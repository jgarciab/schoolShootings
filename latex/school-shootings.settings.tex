\lefthyphenmin=3
\righthyphenmin=2

%% build papers out of todo lists
\usepackage{color}
\newcommand{\todo}[1]{\noindent\textcolor{blue}{{$\Box$ #1}}}

\usepackage{graphicx,epsfig,verbatim,enumerate}
\usepackage{amssymb,amsmath}
\usepackage{ifthen}

\usepackage{longtable}

\usepackage{mathtools}
%% \usepackage{abraces}

%% \usepackage{tikz-cd}
%% \usetikzlibrary{arrows}
%% \tikzset{
%%   commutative diagrams/.cd,
%%   arrow style=tikz,
%%  diagrams={>=space}}

\newboolean{twocolswitch}

\newcommand{\avg}[1]{\left\langle#1\right\rangle}
\newcommand{\tavg}[1]{\langle#1\rangle}

\newcommand{\LavgHa}[2]{\tavg{L_{#1,#2}}}
\newcommand{\LavgHaN}[2]{\tavg{L_{#1,#2}}}
\newcommand{\Lavg}{\tavg{L}}

\newcommand{\sindex}[1]{}
\newcommand{\nindex}[1]{}

\revtexlatexswitch{\newcommand{\etal}{\textit{et al.}}}{}
\newcommand{\www}[1]{\url{#1}}
\newcommand{\req}[1]{(\ref{#1})}
\newcommand{\Req}[1]{Eq.~(\ref{#1})}

% lettrines
\usepackage{lettrine}

\newcommand{\tbf}{\textbf}
\newcommand{\tit}{\textit}

%% differential equations and integrals
\newcommand{\dee}[1]{\mbox{d}#1}
\newcommand{\pdiff}[2]{\frac{\partial #1}{\partial #2}}
\newcommand{\pdiffsq}[2]{\frac{\partial^2 #1}{{\partial #2}^2}}
\newcommand{\diff}[2]{\frac{{\rm d}#1}{{\rm d}#2}}
\newcommand{\diffsq}[2]{\frac{{\rm d}^{2}#1}{{\rm d} {#2}^2}}
\newcommand{\tdiff}[2]{\mbox{d} #1/\mbox{d} #2}
\newcommand{\tdiffsq}[2]{\mbox{d}^{2} #1/\mbox{d} {#2}^2}
\newcommand{\tpdiff}[2]{\partial #1/\partial #2}
\newcommand{\tpdiffsq}[2]{\partial^2 #1/\partial {#2}^2}
\newcommand{\postdee}[1]{\,\mbox{d}#1}


\newcommand{\Prob}[1]{{\rm Pr}\left(#1\right)}

%% contagion

% \newcommand{\kstar}{d^{\ast}}
% \newcommand{\kstari}{k_i^\ast}
\newcommand{\kstar}{d^\ast}
\newcommand{\kstari}{d_i^\ast}

\newcommand{\dstar}{d^\ast}
\newcommand{\dstari}{d_i^\ast}

\newcommand{\phifix}{{\phi^{\ast}}}
\newcommand{\phifixc}{{\phi_c^{\ast}}}
\newcommand{\phifixb}{{\phi_b^{\ast}}}

\newcommand{\phiiactive}{\phi_{i,t}}

\newcommand{\phiiup}{{\phi_{i,\rm on}}}
\newcommand{\phiidown}{{\phi_{i,\rm off}}}

\newcommand{\phiup}{{\phi_{\rm on}}}
\newcommand{\phidown}{{\phi_{\rm off}}}


\newcommand{\Gfun}{G}

\newcommand{\dstardist}{g}
\newcommand{\dosedist}{f}
\newcommand{\dosedistk}{f^{k\star}}

% chaotic contagion
\newcommand{\edgeinfprob}{\rho}
\newcommand{\nodeinfprob}{\phi}

\newcommand{\avgdegree}{k_{\rm avg}}

\newcommand{\stateA}{S_{0}}
\newcommand{\stateB}{S_{1}}

\newcommand{\lcce}{H}

%% \newcommand{\effectivecount}{f_{\textnormal{eff}}}

\newcommand{\effectivecount}{f_{q,\textnormal{exp}}}
\newcommand{\effectivecountq}[1]{f_{#1,\textnormal{exp}}}

\newcommand{\simonrho}{\rho}
\newcommand{\zipfexponent}{\alpha}

\newcommand{\partitionprob}{q}
\newcommand{\partitiontype}[1]{$\partitionprob$$=$$#1$}

\newcommand{\onehalf}{\frac{1}{2}}
\newcommand{\onequarter}{\frac{1}{4}}

\newcommand{\clgamma}{\hat{\zipfexponent}}
\newcommand{\xmin}{r_{\textnormal{max}}}
\newcommand{\kstat}{D}
\newcommand{\clp}{\mbox{$p$-value}}
\newcommand{\sigamma}{1-\simonrho}




%% school shootings

%% Delete for otherfi
%% \usepackage[pdftex]{graphicx}
%% \usepackage{hyperref}

%% \usepackage{scicite}
%% \usepackage{times}
%% \usepackage{parskip}

\newcommand{\beginsupplement}{%
        \setcounter{table}{0}
        \renewcommand{\thetable}{S\arabic{table}}%
        \setcounter{figure}{0}
        \renewcommand{\thefigure}{S\arabic{figure}}%
     }

%% 
%% \renewcommand\refname{References and Notes}
%% 

% The following lines set up an environment for the last note in the
% reference list, which commonly includes acknowledgments of funding,
% help, etc.  It's intended for users of BibTeX or the {thebibliography}
% environment.  Users who are hand-coding their references at the end
% using a list environment such as {enumerate} can simply add another
% item at the end, and it will be numbered automatically.

%% \newcounter{lastnote}
%% \newenvironment{scilastnote}{%
%% \setcounter{lastnote}{\value{enumiv}}%
%% \addtocounter{lastnote}{+1}%
%% \begin{list}%
%% {\arabic{lastnote}.}
%% {\setlength{\leftmargin}{.22in}}
%% {\setlength{\labelsep}{.5em}}}
%% {\end{list}}


%% \makeatletter
%% \renewcommand{\maketitle}{\bgroup\setlength{\parindent}{0pt}
%% \begin{flushleft}
%%   \textbf{\@title}
%% 
%%   \@author
%% \end{flushleft}\egroup
%% }
%% \makeatother
