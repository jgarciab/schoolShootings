Over the past two decades, school shootings within the United States 
have repeatedly devastated communities and shaken public opinion.
Many of these attacks appear to be `lone wolf' ones driven by
specific individual motivations,
and the identification of precursor signals and hence
actionable policy measures would thus seem highly unlikely.
Here, we take a system-wide view and investigate 
the timing of school attacks and the dynamical
feedback with social media.
We identify a trend divergence in which college attacks have continued
to accelerate over the last 25 years 
while those carried out on K-12 schools have slowed down.
We establish the copycat effect in school shootings and uncover a
statistical association between social media chatter
and the probability of an attack in the following days. 
While hinting at causality, this relationship may also help mitigate the
frequency and intensity of future attacks.
